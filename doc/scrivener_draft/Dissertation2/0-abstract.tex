In many real-world environments such as flooded pipes or caves, exteroceptive sensors, such as vision, range or touch, often fail to give any useful information for robotic tasks.  This may result from complete sensor failure or incompatibility with the ambient environment.  We investigate how proprioceptive sensors can help robots to successfully explore, map and navigate in these types of challenging environments.

Our approach is to use a snake robot with proprioceptive joint sensors capable of knowing its complete internal posture.  From this posture over time, we are able to sweep out the free space of confined pipe-like environments.  With the free space information, we incrementally build a map.  The success of mapping is determined by the ability to re-use the map for navigation to user-directed destinations.

We address the following challenges: 1) How does the robot move and locomote in a confined environment without exteroception?  2) How is the distance traveled by a snake robot measured with no odometry and no exteroception?  3) How does the robot sense the environment with only proprioception?  4)  How is the map built with the information available?  5) How is the map corrected for visiting the same location twice, i.e. loop-closing?  6)  How does the robot navigate and explore with the constructed map?

In order to move through the environment, the robot needs to have a solution for motion planning and collision-reaction.  In an exteroceptive approach, we would detect and avoid obstacles at a distance.  If collisions were made, touch sensors could detect them and we could react appropriately.  With only proprioceptive sensors, indirect methods are needed for reacting to obstacles.  A series of motion methods are used to solve these problems including compliant locomotion, compliant path-following, safe-anchoring, stability assurance, slip detection, and dead-end detection.  We show results demonstrating their effectiveness.

While moving through the environment, the snake robot needs some means of measuring the distance traveled.  Wheeled robots usually have some form of shaft encoder to measure rotations of the wheels to estimate distance traveled.  In addition, range or vision sensors are capable of tracking changes in the environment to estimate position of the robot.   GPS is not feasible because of its unreliable operation in underground environments.  Our proprioceptive approach achieves motion estimation by anchoring multiple points of the body with the environment and kinematically tracking the change in distance as the snake contracts and extends.   This gives us a rudimentary motion estimation method whose effectiveness we measure in a series of experiments.

Some method is needed to sense the environment.  In an exteroceptive approach, vision and range sensors would give us a wealth of information about the obstacles in the environment at great distances.  Touch sensors would give us a binary status of contact with an obstacle or not.  With only proprioceptive joint sensors, our approach is to kinematically compute the occupancy of the robot’s body in the environment and record the presence of free space over time.  Using this information, we can indirectly infer obstacles on the boundary of free space.  We show our approach and the sensing results for a variety of environmental configurations.

Combining the multiple snapshots of the local sensed free space, the robot needs a means of building a map.  In the exteroceptive approach, one would use the ability to sense landmark features at a distance and identify and merge their position across several views.  However, in a proprioceptive approach, all sensed information is local.  The opportunities for spotting landmark features between consecutive views are limited.  Our approach is to use the pose graph representation for map-building and add geometric constraints between poses that partially overlap.  The constraints are made from a combination of positional estimates and the alignment of overlapping poses.  We show the quality of maps in a variety of environments.

In the event of visiting the same place twice, we wish to detect the sameness of the location and correct the map to make it consistent.  This is often called the loop-closing or data association problem.  In the case of exteroceptive mapping, this is often achieved by finding a correspondence between sets of landmark features that are similar and then merging the two places in the map.  In the proprioceptive case, the availability of landmark features is scarce.   We instead develop an approach that exploits the properties of confined environments and detects loop-closing events by comparing local environmental topologies.  We show the results of loop-closing events in a variety of environmental junctions.

Once we have constructed the map, the next step is to use the map for exploration and navigation purposes.  In the exteroceptive case, navigating with the map is a localization problem, comparing the sensed environment with the mapped one.  This is also the approach in the proprioceptive case.  However, the localization accuracy along the length of a followed path is more uncertain, so our path-following algorithms are necessarily more robust to this eventuality.   In the exteroceptive case, the act of exploration can be achieved by navigating to the boundaries of the map with no obstacles.  In the proprioceptive case, exploration is achieved by following every tunnel or path until a dead-end is detected.  We show some environments that we successfully map and navigate, as well as some environments that require future work. 


